%Preambulo---------------------------------------------------------------------------------------
\documentclass[11pt,a4paper]{article}
\usepackage[spanish]{babel}
\usepackage{graphicx}
\usepackage[margin=2.54cm]{geometry}
\usepackage[utf8]{inputenc}
\usepackage{ragged2e}
\usepackage[font = footnotesize,labelfont=it]{caption}


\pagestyle{empty}
%fin del preambulo-------------------------------------------------------------------------------

%Fin del preambulo-------------------------------------------------------------
\begin{document}
\begin{center}
{\textbf{\large Procesamiento de Señales Utilizando Herramientas Matemáticas y Computacionales para su Análisis\\}}
{\footnotesize Universidad Nacional Autónoma de México\\
Facultad de Estudios Superiores Cuautitlán\\} 
{\footnotesize Alonso Vargas Gachuz}
\end{center}

\begin{abstract}
La inteligencia artificial necesita tener acceso a una gran cantidad de información de la cual puede devolver una conclusión útil que podamos interpretar, en la mayoría de los casos la información obtenida no es apta para trabajarla, esta puede tener valores complejos o abstractos para el sistema imposibilitando la llegada a una conclusión. El preprocesamiento de la información es el primer paso para el desarrollo de sistemas inteligentes y quizá sea el más importante ya que aquí podemos encontrar y definir las características que nos interese averiguar. En este documento se relata el proceso para analizar y dar tratamiento de una base de datos la cual registra las vibraciones en los motores de ventilación con el fin de predecir una posible falla bajo diferentes condiciones. Se adecuaron los datos para su posterior uso en algún sistema. Para el procesamiento se proponen los siguientes pasos: limpiar y corregir datos faltantes, normalización, determinación de elementos estadísticos, determinación de cruces a cero, cambios de signo y aplicación de algoritmos para preprocesamiento. Con el fin de cumplir con estas tareas se usaron las herramientas Matlab y Python. Python fue usado para la limpieza y corrección de los datos mientras que Matlab se usó para el procesamiento de las señales y por su capacidad de realizar operaciones matemáticas.
\end{abstract}

\section{Introducción}
El diseño de cualquier sistema o dispositivo debe prever su uso y comportamiento en cualquier ambiente ya sea que existan condiciones inadecuadas o un mal manejo de estos. Un buen diseño debe contemplar estos aspectos y es por ello que se deben realizar pruebas en los dispositivos para conocer los limites de estos y prever el fallo en estos o advertir las condiciones en las que se deben operar. El proceso de prueba puede ser demasiado tardado si no se dispone de un antecedente o en caso de no saber por donde empezarlas. Para poder probar esta clase de motores se propone el uso de un sistema de predicción basado en un análisis previo, este sistema debe tener la capacidad de recolectar los datos y predecir el tiempo de falla de un motor según los parámetros de entrada. Como un primer paso para la creación de dicho sistema se debe emplear una base de datos de la cual dispongamos de la información necesaria para realizar el análisis adecuado. Con ayuda de una red neuronal es posible crear un sistema potente capaz de predecir la vida útil restante de motores en uso, ya sea para prevenir fallas venideras e incluso para el análisis en el diseño de dispositivos que requieran el uso de motores con tal de hacer dispositivos más robustos que soporten mejor las vibraciones evitando daños importantes.
\section{Conocimientos Previos}
\textbf{Transformada de Fourier:} La Transformada de Fourier es una operación matemática indispensable para un gran número de disciplinas. Se usa en campos como la medicina, las telecomunicaciones, la ingeniería acústica, los circuitos eléctricos, el diseño de puentes frente a resonancias y la compresión de pistas de audio, entre otros.

\textbf{Transformada Wavelet:} Las wavelets son señales, o formas de onda, las cuales tienen una duración limitada y un valor promedio de cero. Las wavelets pueden ser irregulares y asimétricas, características que les otorgan una mejor adaptación en el análisis de señales en comparación con la transformada de Fourier.\cite{intar}
\section{Metodología}
Con el fin de tratar los datos de las señales para que puedan ser interpretados por el sistema se deben realizar pasos de normalización y análisis de la información. Como primer paso se debe hacer una limpieza de los datos, normalización, búsqueda de información por medio de estadística descriptiva, búsqueda de picos y cruces a cero. A continuación se describe el método para el tratamiento de la base de datos. Dado que se empleó Matlab y Python con Jupyter los códigos del proceso descrito se encuentran en el repositorio de GitHub: \url{}
\subsection{Limpieza de datos}
Para verificar que no existan datos faltantes se utilizó el lenguaje de programación Python con la librería "Pandas" la cual nos permite trabajar con vectores y sus registros además de realizar cambios y búsquedas en la información. Con ayuda de la plataforma Jupyter y trabajando en Python podemos visualizar la información, es necesario añadirlas librerías de pandas, numpy y os para trabajar adecuadamente. Posteriormente se crea una instancia donde se guardará la base de datos empleada.
\section{Análisis de Resultados}
Aquí va el Análisis de Resultados
\section{Conclusiones}
Aquí van las conclusiones

\bibliographystyle{ieeetr}
\begin{thebibliography}{x}
	\bibitem{intar} \textit{Transformada Wavelet - acervo para el mejoramiento del aprendizaje de alumnos de ingeniería, en Inteligencia Artificial}.(s.f). \url{https://virtual.cuautitlan.unam.mx/intar/?page\_id=1108}
	
	\bibitem{fourier} \textit{La transformada de Fourier}
	
	\bibitem{matlab} \textit{el uso de matlab}
	
	\bibitem{accel_cite} \textsc{Scalabrini Sampaio,Gustavo, Rabello de Aguiar Vallim Filho,Arnaldo, Santos de Silva,Leilton, and Augusto da Silva,Leandro}. (2023). \textit{Accelerometer}. UCI Machine Learning Repository. \url{https://doi.org/10.24432/C5Q61V}.
\end{thebibliography}

\end{document}